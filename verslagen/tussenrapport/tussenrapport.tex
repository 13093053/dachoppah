%% LyX 2.1.0 created this file.  For more info, see http://www.lyx.org/.
%% Do not edit unless you really know what you are doing.
\documentclass[english]{article}
\usepackage[T1]{fontenc}
\usepackage[latin9]{inputenc}
\setlength{\parskip}{\bigskipamount}
\setlength{\parindent}{0pt}
\usepackage{textcomp}
\usepackage{babel}
\begin{document}

\title{Tussenrapport PROENT-pj1\\
Elektrotechniek}


\author{Groep2\\
Andrew Lau\\
Daan Conijn\\
Dani�l Martoredjo\\
Kevin Oei\\
Koen van Vliet\\
Wouter Boogert}


\date{28 november 2014}

\maketitle
$\pagebreak$

\tableofcontents{}

$\pagebreak$


\section*{Inleiding}

Het is de bedoeling dat lokaal 0.126 wordt ingericht voor digitale
toetsen. Hiervoor moet het lokaal van 100 PC's worden voorzien. Wij
onderzochten op welke manieren dit te bewerkstelligen is. Hierin beperkten
wij ons tot de stroomvoorziening en de data-verbinding van de PC's.
Ook vergeleken we desktop PC's met M2Desk ge�ntegreerde PC's.

$\pagebreak$


\section{Opstelling van tafels}

We adviseren een opstelling bestaande uit 10 rijen van 10 computers.
Gezien er maximaal tien tafels in de breedte passen in het lokaal
zonder dat er wordt afgeweken van de richtlijnen voor toetsopstellingen,
is dit een realiseerbare opstelling. De kabel voor de voeding van
de PC's afkomstig van ��n zekeringsgroep loopt recht van de ene kant
van de opstelling naar de ander. Op deze manier kan er een gelijk
aantal computers worden aangesloten per groep zonder dat de groepen
maximaal belast worden. Deze opstelling is voor alle scenarios (Desktop
PC of M2Desk, AC of DC) hetzelfde.

$\pagebreak$


\section{Stroomvoorziening}

Elke kolom wordt van stroom voorzien door een kabel die vertakt bij
elke PC in de kolom. De kabels liggen naast de tafels zodat ze niet
in de weg liggen. Hierop kunnen maximaal 10 computers worden aangesloten.
Er zijn dus minimaal 10 groepen nodig. Het is belangrijk dat deze
groepen niet aan de aardlek geschakeld zijn in verband met lekstromen
in de computers. De lekstromen zijn dusdanig hoog dat de aardlek afschakelt. 


\subsection{Desktop PC AC 230V}

Per groep loopt er ongeveer een stroom van 13A (10 {*} 300/230). Elke
kolom wordt aangesloten op een aparte groep gezekerd op 16A. De kabels
moeten minimaal 16A aankunnen en voldoen aan de Nederlandse veiligheidsnorm
voor laagspanningsinstallaties. De draaddikte is minimaal 2,5mm\textsuperscript{2}.
De spanningsval als gevolg van draadweerstand is verwaarloosbaar met
de korte draadlengtes in deze installatie.


\subsection{Desktop PC DC 350V}

Voor een DC installatie kunnen dezelfde kabels worden gebruikt. Door
de hogere spanning kunnen er meer computers per groep worden aangesloten.
Als er nog steeds 10 computers op een groep worden aangesloten kunnen
er ook dunnere kabels worden gebruikt en zekeringen van 10A. Zo kan
de installatie goedkoper worden gemaakt. Door de hogere spanning wordt
de installatie ook effici�nter. 


\subsection{M2Desk AC 230V}

De M2Desk computers verbruiken veel minder stroom dan de normale 200-300W
PC's. De M2Desk computers vereisen samen een stroomsterkte van minder
dan 11A. Het is mogelijk om alles op een groep van 16A aan te sluiten.
Hiervoor kunnen dezelfde 2,5mm\textsuperscript{2}kabels worden gebruikt.
Om elke kolom aan te sluiten kan een dunnere kabel gebruikt worden,
omdat deze slechts 1,1A per stuk hoeven te leveren.


\subsection{M2Desk DC 350V}

Bij een spanning van 350V is het mogelijk om nog dunnere kabels te
gebruiken, omdat er minder stroom geleverd hoeft te worden. Alle computers
kunnen worden aangesloten op een groep gezekerd op 10A. Verliezen
in kabels worden ook kleiner door deze hogere spanning.

$\pagebreak$


\section{LAN data-verbinding}

Iedere computer moet een verbinding met het lokale netwerk hebben.
De datavoorziening moet stabiel, veilig en redelijk snel zijn. Het
is belangrijk dat het netwerk niet zomaar stil komt te liggen tijdens
een toets, omdat de toets dan opnieuw moet worden afgenomen op een
ander moment.


\subsection{Draadloos (WIFI)}

Dit type netwerk is zonder extra kabels te realiseren waardoor het
snel op te zetten is. Een draadloze verbinding brengt wel een beveiligingsrisico
met zich mee. Het is mogelijk om met een ander netwerk te verbinden
zoals Eduroam. Dit netwerk heeft vrije internet toegang. Dit is niet
wenselijk in een toetsopstelling.


\subsection{Data via de voedingskabels}

Dit is ook zonder extra kabels te realiseren, maar het is nog maar
de vraag of het mogelijk is om 100 PC's op dit type netwerk aan te
sluiten. 100 PC voedingen zorgen voor veel verstoring op de kabels.
Elke computer moet bovendien voorzien worden van een apparaatje wat
het datasignaal uit de stroomvoorziening extraheert. Dit brengt extra
kosten met zich mee.


\subsection{Ethernetkabels}

Iedere PC wordt voorzien van een eigen ethernetkabel. Het kost wat
extra tijd om het netwerk aan te leggen, maar dit garandeert het meest
stabiele, veilige en snelste netwerk.

$\pagebreak$


\section{Opbouwtijd}


\subsection{Desktop PC}

Er zijn veel verschillende soorten desktop PC's. Hier wordt als voorbeeld
hetzelfde type PC gebruikt als bij het studielandschap elektrotechniek
gebruikt. Het neerzetten van de tafels kost ongeveer 3-4 uur. Het
kost ongeveer 2-3 uur om elke tafel van een PC te voorzien. Er moeten
ook nog toetsenborden en muizen worden aangesloten. Dit kost ongeveer
2 uur extra. Daarna moeten de PC's nog van stroom en e.v.t. ethernetkabels
worden voorzien. Het uitrollen en aansluiten van de bedrading kost
ongeveer 2-3 uur. In totaal kost het 9-12 uur om alle PC's neer te
zetten en startklaar te maken. Bij een uurtarief van 50 euro per uur
kost het naar schatting 450-600 euro om alles in orde te maken.


\subsection{M2Desk}

Het kost een werkdag om 100 M2Desk computers neer te zetten. Bij een
uurtarief van 50 euro per uur kost het ongeveer 450 euro om alle computers
te installeren. De M2Desk computers kunnen worden getransformeerd
tot normale tafels, dus deze opstelling kan blijven staan voor alle
toetsen.

$\pagebreak$


\section{Onderhoud}

Het komt voor dat de ethernet- en stroomkabels stuk gaan waardoor
de kabels niet meer goed functioneren. De kapotte kabels moeten dan
vervangen worden. De prijs per meter is ongeveer 2 euro. Het is gekozen
voor vervangen van kapotte kabels en niet voor regelmatig controles,
omdat het vervangen van kapotte kabels goedkoper is dan regelmatig
controles.

$\pagebreak$


\section*{Conclusie}

Wij bevelen aan om M2Desk geintegreerde PC's in combinatie met een
DC netspanning en een aparte ethernetkabel te gebruiken. Hoewel het
opstellen van de tafels in het geval van zowel desktop PC's als M2Desks
hetzelfde is, zullen de tafels bij desktop PC's sneller schade oplopen
door slijtage door het continu neerzetten en opruimen van de PC opstellingen.
Op de langer termijn kan het daardoor aantrekkelijker zijn om met
M2Desk opstellingen te werken in plaats van normale tafels in combinatie
met desktop PC's. Deze computers verbruiken zeer weinig energie vergeleken
met normale desktop PC's. Hierdoor worden de energiekosten lager maar
ook de kostprijs van de bekabeling. Een aparte ethernetkabel per PC
mag dan wel duurder zijn, maar het garandeert wel het meest stabiele
netwerk. Het opbouwen van de toetsopstelling is sneller met M2Desks
dan met desktop PC's en bovendien kunnen er normale toetsen worden
gehouden in een lokaal ingericht met M2Desk computers.


\section*{Bijlagen}

Alle gebruikte tabellen komen uit de NEN1010.\\
Montage-methode E (tabel A.52-1) type kabel (tabel A.52-2) (H07RN-F
slijtvaste mantel, bestand tegen zware mechanische belasting)


\subsubsection*{Berekening draad voor 10 normale PC's AC}

Stap 1: De bedrijfstroom is: (300/230) {*} 10 = 13A\\
Stap 2: De eerste standaardwaarde boven 13A is 16A.

Als zekering kiezen we voor een 16A trage diazed patroon. Op een automaat
kunnen maximaal 5 standaard computers door de hoge inschakelstromen.
Een traag smeltpatroon kan een veel hogere inschakelstroom verdragen. 

Stap 3: De bijbehorende Iz van de leiding is bij dit type zekering
17,7A.\\
Stap 4: De correctiefactor voor de omgevingstemperatuur van 30\textdegree C
is 1,00 (tabel A.52-15).\\
Stap 5: De Iz is 17,7A/1,0 = 17,7A de doorsnede wordt dan 1,5mm\textsuperscript{2}
(tabel A.52-9) . De weerstand van deze draad is 15.4 ohm/km.\\
Stap 6: De maximale lengte is 53m (tabel A.53-4).De lengte van het
lokaal is 30m, dus dit is voldoende.

De maximale spanningsval bij de nominale stroom is: 13{*}((15,4/1000){*}30)
= 6V


\subsubsection*{Berekening draad voor 10 normale PC's DC}

Stap 1: De bedrijfstroom is: (300/350) {*} 10 = 8,6A\\
Stap 2: De eerste standaardwaarde boven 8,6A is 10A.

Als zekering kiezen we voor een 10A trage diazed patroon. Op een automaat
kunnen maximaal 5 standaard computers door de hoge inschakelstromen.
Een traag smeltpatroon kan een veel hogere inschakelstroom verdragen. 

Stap 3: De Iz van de leiding kan gelijk worden gesteld aan de nominale
stroom van de automaat. Dit is dus 10A.\\
Stap 4: De correctiefactor voor de omgevingtemperatuur van 30\textdegree C
is 1,00 (tabel A.52-15).\\
Stap 5: De Iz is 13,1A/1,0 = 13,1A de doorsnede wordt dan 1,5mm\textsuperscript{2}
(tabel A.52-9) . De weerstand van deze draad is 15.4 ohm/km.\\
Stap 6: De maximale lengte is 53m (tabel A.53-4).De lengte van het
lokaal is 30m, dus dit is voldoende.

De maximale spanningsval bij de nominale stroom is: 8,6{*}((15,4/1000){*}30)
= 4V


\subsubsection*{Berekening draad voor M2Desk PC's AC}

Stap 1: De bedrijfsstroom voor 100 M2Desk PC's is (2400/230) = 10,4A\\
Stap 2: De eerste standaardwaarde boven 10,4A is 16A

Als zekering kiezen we voor een 16A trage diazed patroon. Een automaat
kan de hoge inschakelstromen van de PC's niet verdragen. Een traag
smeltpatroon wel.

Stap 3: De bijbehorende Iz van de leiding is bij dit type zekering
17,7A.\\
Stap 4: De correctiefactor voor de omgevingtemperatuur van 30\textdegree C
is 1,00 (tabel A.52-15).\\
Stap 5: De Iz is 17,7A/1,0 = 17,7A de doorsnede wordt dan 1,5mm\textsuperscript{2}
(tabel A.52-9) . De weerstand van deze draad is 15.4 ohm/km.\\
Stap 6: De maximale lengte is 53m (tabel A.53-4).De lengte van het
lokaal is 30m, dus dit is voldoende.

De maximale spanningsval bij de nominale stroom is: 10,4{*}((15,4/1000){*}30)
= 4,8V


\subsubsection*{Berekening draad voor M2Desk PC's DC}

Stap 1: De bedrijfsstroom voor 100 M2Desk PC's is (2400/350) = 6,8A\\
Stap 2: De eerste standaardwaarde boven 6,8A is 10A.

Als zekering kiezen we voor een 10A trage diazed patroon. Een automaat
kan de hoge inschakelstromen van de PC's niet verdragen. Een traag
smeltpatroon wel.

Stap 3: De bijbehorende Iz van de leiding is bij dit type zekering
13,1A\\
Stap 4: De correctiefactor voor de omgevingtemperatuur van 30\textdegree C
is 1,00 (tabel A.52-15).\\
Stap 5: De Iz is 13,1A/1,0 = 13,1A de doorsnede wordt dan 1,5mm\textsuperscript{2}
(tabel A.52-9) . De weerstand van deze draad is 15.4 ohm/km.\\
Stap 6: De maximale lengte is 53m (tabel A.53-4).De lengte van het
lokaal is 30m, dus dit is voldoende.

De maximale spanningsval bij de nominale stroom is: 6,8{*}((15,4/1000){*}30)
= 3,1V
\end{document}
